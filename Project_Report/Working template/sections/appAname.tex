\chapter{Appendix A : \arduino{} Code}\label{ch:appAlabel}
Here is the code uploaded on to the \arduinouno{}:
\begin{mdframed}[backgroundcolor=light-gray, roundcorner=10pt,leftmargin=1, rightmargin=1, innerleftmargin=15, innertopmargin=15,innerbottommargin=15, outerlinewidth=1, linecolor=light-gray]
\begin{lstlisting}[caption={The Arduino Code},language = C]
/*ver 1.1 (Final)*/

// Includes the Servo library
#include <Servo.h>. 

// Defines Trig and Echo pins of the Ultrasonic Sensor
const int trigPin = 2;
const int echoPin = 4;

// Variables for the duration & distance measured from the sensor
float duration;
float distance;

//Angle offset between graph and motor's XY axes
int offset = 0;
//Initial and final angles of motor's rotation
//Note that the motor can only take values 0-180.
//Make sure init_angle-offset >=60 to avoid jumper wire obstruction.
int init_ang = 70+offset;
int fin_ang = 180+offset;

// Declares a structure variable/object of the type Servo,
//(defined in the servo library) for controlling the servo motor.
Servo myServo; 

void setup() {
pinMode(trigPin, OUTPUT); // Sets the trigPin as an Output
pinMode(echoPin, INPUT); // Sets the echoPin as an Input
Serial.begin(9600); //Sets baud rate of serial communication
myServo.attach(9); // Defines on which pin is the servo motor attached
}

void loop() {
// if (Serial.available() > 0)  /*piece of code to be included in case of serial communication with IoT board*/
// {
// rotates the servo motor from init_ang to fin_ang degrees
for(int i=init_ang;i<=fin_ang;i++)
{  
myServo.write(i-offset);  //angle value to be passed to the servo library object for writing into the motor
delay(17);//DELAY #1:for time taken in motor rotation for one degree before calculating distance
distance = calculateDistance();// Calls a function for calculating the distance measured by the Ultrasonic sensor for each degree
Serial.print(i); // Sends the current degree into the Serial Port for graphical representation
Serial.print(","); // Sends addition character right next to the previous value needed later in the Processing IDE for indexing
Serial.print(distance); // Sends the distance value into the Serial Port for the graph
Serial.print(";"); // Sends addition character right next to the previous value needed later in the Processing IDE for indexing
}
// Repeats the previous lines from fin_ang to init_ang degrees
for(int i=fin_ang;i>init_ang;i--)
{  
myServo.write(i);
delay(17);  //DELAY #1 //can be minimised to 17 or 1667 microsec
distance = calculateDistance();
Serial.print(i);
Serial.print(",");
Serial.print(distance);
Serial.print(";");
}
//}
}
// Function for calculating the distance measured by the Ultrasonic sensor
float calculateDistance(){ 
unsigned long T1 = micros();
digitalWrite(trigPin, LOW); // trigPin needs a fresh LOW pulse before sending a HIGH pulse that can be detected from echoPin
delayMicroseconds(2);//DELAY #2:time for which low trig pulse is maintained before making it high
digitalWrite(trigPin, HIGH); 
delayMicroseconds(10);//DELAY #3:Sets the trigPin on HIGH state for 10 micro seconds
digitalWrite(trigPin, LOW);
duration = pulseIn(echoPin, HIGH); // Reads the echoPin, returns the sound wave travel time in microseconds
//distance= duration*0.034/2;
distance = (duration/2)/29.1;     //in cm,  datasheet gives "duration/58" as the formula

//To avoid sending data at variable time intervals due to varying time duration taken between execution of above code inside this function depending on distance of obstacle
//if no object, echo pulse is HIGH after 38ms
while(micros()-T1<38000)
{
;
}

return distance;
}
\end{lstlisting}

\end{mdframed}
\clearpage

\chapter{Appendix B : \processing{} Code}\label{ch:appBlabel}
The \emph{Java} code that we ran in \processing{} is given below :
\begin{mdframed}[backgroundcolor=light-gray, roundcorner=10pt,leftmargin=1, rightmargin=1, innerleftmargin=15, innertopmargin=15,innerbottommargin=15, outerlinewidth=1, linecolor=light-gray]
\begin{lstlisting}[caption={The Processing Code}, language=Java]
/*   Arduino Radar Project
*    1.1 [Final and complete]
*   Initialise every variable to null value to avoid null pointer exception
*/

import processing.serial.*; // imports library for serial communication
import java.awt.event.KeyEvent; // imports library for reading the data from the serial port
import java.io.IOException;

Serial myPort; // defines Object Serial
// defines variables
String angle="";
String distance="";
String data="";
String noObject="";
float pixsDistance=0.0;
int iAngle=0;
float iDistance=0.0;
int index1=0;
int index2=0;
int objCount=0,obctr=0;
PFont orcFont;
float angFlag=1.0;
float prevAng=0.0,deltaAng=0.0;
float maxDIST=50.0; //max distance of object in cms
int wctr=0;
float objWidth=0.0,prevWidth=0.0;
void setup() {

size (600, 700); // ***CHANGE THIS TO YOUR SCREEN RESOLUTION***
smooth();

myPort = new Serial(this,"/dev/ttyACM0", 9600); // Enter the COM Port address as COM4 or COM 22.starts the serial communication

myPort.bufferUntil('.'); // reads the data from the serial port up to the character '.'. So actually it reads this: angle,distance.
orcFont = loadFont("CenturySchL-Ital-20.vlw");

}

void draw() {

fill(237,13,245);
textFont(orcFont);
// simulating motion blur and slow fade of the moving line
noStroke();
fill(184,13); 
rect(0, 0, width, height-height*0.065); 

fill(211,27,228); //  color
// calls the functions for drawing the radar
drawRadar(); 
drawLine();
drawObject();
drawText();
}

void serialEvent (Serial myPort) { // starts reading data from the Serial Port
// reads the data from the Serial Port up to the character '.' and puts it into the String variable "data".
try{
data = myPort.readStringUntil(';');
data = data.substring(0,data.length()-1);

index1 = data.indexOf(","); // find the character ',' and puts it into the variable "index1"
angle= data.substring(0, index1); // read the data from position "0" to position of the variable index1 or thats the value of the angle the Arduino Board sent into the Serial Port
distance= data.substring(index1+1, data.length()); // read the data from position "index1" to the end of the data pr thats the value of the distance

// converts the String variables into Integer
iAngle = int(angle);
iDistance = float(distance);
deltaAng=iAngle-prevAng;

if(deltaAng*angFlag<0.0){
objCount=0;
objWidth=0;
}
//anticlockwise--deltaAng>0
//clockwise--deltaAng<0    
if(deltaAng>0.0)
angFlag=1.0;
else
angFlag=-1.0;
}
catch(Exception e){
println("Error parsing:");
e.printStackTrace();
}
}

void drawRadar() {
pushMatrix();
translate(width/2,height-height*0.507); // moves the starting coordinates to new location
noFill();
strokeWeight(2);
stroke(407,409,305);
// draws the arc lines
arc(0,0,(width-width*0.0385),(width-width*0.0385),PI,TWO_PI);
arc(0,0,(width-width*0.26),(width-width*0.26),PI,TWO_PI);
arc(0,0,(width-width*0.498),(width-width*0.498),PI,TWO_PI);
arc(0,0,(width-width*0.754),(width-width*0.754),PI,TWO_PI);
arc(0,0,(width-width*0.0385),(width-width*0.0385),0,PI);
arc(0,0,(width-width*0.26),(width-width*0.26),0,PI);
arc(0,0,(width-width*0.498),(width-width*0.498),0,PI);
arc(0,0,(width-width*0.754),(width-width*0.754),0,PI);
// draws the angle lines
line(-width/2,0,width/2,0);
line(0,0,(-width/2)*cos(radians(30)),(-width/2)*sin(radians(30)));
line(0,0,(-width/2)*cos(radians(60)),(-width/2)*sin(radians(60)));
line(0,0,(-width/2)*cos(radians(90)),(-width/2)*sin(radians(90)));
line(0,0,(-width/2)*cos(radians(120)),(-width/2)*sin(radians(120)));
line(0,0,(-width/2)*cos(radians(150)),(-width/2)*sin(radians(150)));
line(0,0,(-width/2)*cos(radians(180)),(-width/2)*sin(radians(180)));
line(0,0,(-width/2)*cos(radians(210)),(-width/2)*sin(radians(210)));
line(0,0,(-width/2)*cos(radians(240)),(-width/2)*sin(radians(240)));
line(0,0,(-width/2)*cos(radians(270)),(-width/2)*sin(radians(270)));
line(0,0,(-width/2)*cos(radians(300)),(-width/2)*sin(radians(300)));
line(0,0,(-width/2)*cos(radians(330)),(-width/2)*sin(radians(330)));
line((-width/2)*cos(radians(30)),0,width/2,0);
popMatrix();
}

void drawLine() {
pushMatrix();
strokeWeight(8);
stroke(32,65,174); //color for blue line
translate(width/2,height-height*0.507); // moves the starting coordinates to new location
line(0,0,(height-height*0.575)*cos(radians(iAngle)),-(height-height*0.575)*sin(radians(iAngle))); // draws the line according to the angle
popMatrix();
}

void drawObject() {
pushMatrix();
translate(width/2,height-height*0.508); // moves the starting coordinats to new location
strokeWeight(8);
stroke(240,1,40); // red color
pixsDistance = iDistance*((height-height*0.1666)*0.013/3); // covers the distance from the sensor from cm to pixels
if(iDistance<=maxDIST){
if(wctr>2){    //Assuming that an object will be thick enough to be detected for 2 degrees of rotation.
objWidth=0.0;
// draws the object according to the angle and the distance
line(pixsDistance*cos(radians(iAngle)),-pixsDistance*sin(radians(iAngle)),(width-width*0.505)*cos(radians(iAngle)),-(width-width*0.505)*sin(radians(iAngle)));
obctr++;
}
}
popMatrix();
}

void drawText() { // draws the texts on the screen
pushMatrix();
if(iDistance>maxDIST) {
noObject = "No object within Range";
if(wctr==0)
objWidth=prevWidth;
else
objWidth=float(wctr)*iDistance*0.0174;  //width of object in cm
prevWidth=objWidth;
wctr=0;
if(obctr>0){
objCount++;
obctr=0;
}
}
else {
wctr++;
if(wctr>2)  //Assuming that an object will be thick enough to be detected for 2 degrees of rotation.
{
noObject = "Object in Range";

}
} 

fill(0,0,0);  //black background of bottom text
noStroke();
rect(0, height-height*0.0521, width, height);
fill(251,255,249);
textSize(15);

text("30cm",width-width*0.4041,height-height*0.4793);
text("60cm",width-width*0.281,height-height*0.4792);
text("90cm",width-width*0.177,height-height*0.4792);
text("120cm",width-width*0.0729,height-height*0.4792);
textSize(16);
text(noObject, width-width*0.634, height-height*0.0218);
text("Angle: " + iAngle +" @\degree@", width-width*0.97, height-height*0.0232);
text("Distance: ", width-width*0.26, height-height*0.0235);
textSize(16);
text(" No. of objects: "+ objCount +"", width-width*0.986, height-height*0.0714);
text(" Object Width: "+ objWidth +"", width-width*0.986, height-height*0.1714);
if(iDistance<=maxDIST) {
if(wctr>2)
text("        " + iDistance + " cm", width-width*0.185, height-height*0.0237);
}
textSize(19);
fill(7,7,6); //color for degrees text
translate((width-width*0.5020)+width/2*cos(radians(30)),(height-height*0.5283)-width/2*sin(radians(30)));
rotate(-radians(-60));
text("30@\degree@",0,0);
resetMatrix();
translate((width-width*0.507)+width/2*cos(radians(60)),(height-height*0.5139)-width/2*sin(radians(60)));
rotate(-radians(-29));
text("60@\degree@",-4,-5);
resetMatrix();
translate((width-width*0.507)+width/2*cos(radians(90)),(height-height*0.5149)-width/2*sin(radians(90)));
rotate(radians(0));
text("90@@\degree@@",-5,-2);
resetMatrix();
translate(width-width*0.513+width/2*cos(radians(120)),(height-height*0.51286)-width/2*sin(radians(120)));
rotate(radians(-30));
text("120@\degree@",0,0);
resetMatrix();
translate((width-width*0.5298)+width/2*cos(radians(150)),(height-height*0.4803)-width/2*sin(radians(150)));
rotate(radians(-60));
text("150@\degree@",0,0);
resetMatrix();
translate(width-width*0.475+width/2*cos(radians(180)),(height-height*0.47791)-width/2*sin(radians(180)));
rotate(radians(-90));
text("180@\degree@",0,0);
resetMatrix();
translate(width-width*0.494+width/2*cos(radians(210)),(height-height*0.4797)-width/2*sin(radians(210)));
rotate(radians(-120));
text("210@\degree@",0,0);
resetMatrix();
translate(width-width*0.484+width/2*cos(radians(240)),(height-height*0.4867)-width/2*sin(radians(240)));
rotate(radians(-150));
text("240@\degree@",0,0);
resetMatrix();
translate(width-width*0.474+width/2*cos(radians(270)),(height-height*0.50151)-width/2*sin(radians(270)));
rotate(radians(-180));
text("270@\degree@",0,0);
resetMatrix();
translate(width-width*0.470+width/2*cos(radians(300)),(height-height*0.51167)-width/2*sin(radians(300)));
rotate(radians(-210));
text("300@\degree@",0,0);
resetMatrix();
translate(width-width*0.478+width/2*cos(radians(330)),(height-height*0.5174)-width/2*sin(radians(330)));
rotate(radians(-240));
text("330@\degree@",0,0);
resetMatrix();
translate(width-width*0.523+width/2*cos(radians(360)),(height-height*0.52132)-width/2*sin(radians(360)));
rotate(radians(-270));
text("0@\degree@",0,0);
popMatrix(); 
prevAng = iAngle;
}
\end{lstlisting}
\end{mdframed}
\clearpage